%% Generated by Sphinx.
\def\sphinxdocclass{article}
\documentclass[letterpaper,10pt,english]{sphinxhowto}
\ifdefined\pdfpxdimen
   \let\sphinxpxdimen\pdfpxdimen\else\newdimen\sphinxpxdimen
\fi \sphinxpxdimen=.75bp\relax

\PassOptionsToPackage{warn}{textcomp}
\usepackage[utf8]{inputenc}
\ifdefined\DeclareUnicodeCharacter
% support both utf8 and utf8x syntaxes
\edef\sphinxdqmaybe{\ifdefined\DeclareUnicodeCharacterAsOptional\string"\fi}
  \DeclareUnicodeCharacter{\sphinxdqmaybe00A0}{\nobreakspace}
  \DeclareUnicodeCharacter{\sphinxdqmaybe2500}{\sphinxunichar{2500}}
  \DeclareUnicodeCharacter{\sphinxdqmaybe2502}{\sphinxunichar{2502}}
  \DeclareUnicodeCharacter{\sphinxdqmaybe2514}{\sphinxunichar{2514}}
  \DeclareUnicodeCharacter{\sphinxdqmaybe251C}{\sphinxunichar{251C}}
  \DeclareUnicodeCharacter{\sphinxdqmaybe2572}{\textbackslash}
\fi
\usepackage{cmap}
\usepackage[T1]{fontenc}
\usepackage{amsmath,amssymb,amstext}
\usepackage{babel}
\usepackage{times}
\usepackage[Sonny]{fncychap}
\ChNameVar{\Large\normalfont\sffamily}
\ChTitleVar{\Large\normalfont\sffamily}
\usepackage{sphinx}

\fvset{fontsize=\small}
\usepackage{geometry}

% Include hyperref last.
\usepackage{hyperref}
% Fix anchor placement for figures with captions.
\usepackage{hypcap}% it must be loaded after hyperref.
% Set up styles of URL: it should be placed after hyperref.
\urlstyle{same}

\addto\captionsenglish{\renewcommand{\figurename}{图}}
\addto\captionsenglish{\renewcommand{\tablename}{表}}
\addto\captionsenglish{\renewcommand{\literalblockname}{列表}}

\addto\captionsenglish{\renewcommand{\literalblockcontinuedname}{续上页}}
\addto\captionsenglish{\renewcommand{\literalblockcontinuesname}{下页继续}}
\addto\captionsenglish{\renewcommand{\sphinxnonalphabeticalgroupname}{Non-alphabetical}}
\addto\captionsenglish{\renewcommand{\sphinxsymbolsname}{符号}}
\addto\captionsenglish{\renewcommand{\sphinxnumbersname}{Numbers}}

\addto\extrasenglish{\def\pageautorefname{页}}

\setcounter{tocdepth}{3}
\setcounter{secnumdepth}{3}

    \hypersetup{unicode=true}
    \usepackage{CJKutf8}
    \DeclareUnicodeCharacter{00A0}{\nobreakspace}
    \DeclareUnicodeCharacter{2203}{\ensuremath{\exists}}
    \DeclareUnicodeCharacter{2286}{\ensuremath{\subseteq}}
    \DeclareUnicodeCharacter{2713}{x}
    \DeclareUnicodeCharacter{27FA}{\ensuremath{\Longleftrightarrow}}
    \DeclareUnicodeCharacter{221A}{\ensuremath{\sqrt{}}}
    \DeclareUnicodeCharacter{221B}{\ensuremath{\sqrt[3]{}}}
    \DeclareUnicodeCharacter{2295}{\ensuremath{\oplus}}
    \DeclareUnicodeCharacter{2297}{\ensuremath{\otimes}}
    \begin{CJK}{UTF8}{gbsn}
    \AtEndDocument{\end{CJK}}
    \usepackage[table]{xcolor}
    \usepackage{color}
    \usepackage{draftwatermark}
    \SetWatermarkAngle{45}
    \SetWatermarkColor{gray!80!cyan!30}
    \SetWatermarkFontSize{3.5cm}
    \SetWatermarkText{CONFIDENTIAL}

    

\title{rftest Documentation}
\date{2019 年 01 月 28 日}
\release{}
\author{espressif}
\newcommand{\sphinxlogo}{\vbox{}}
\renewcommand{\releasename}{}
\makeindex
\begin{document}

\ifdefined\shorthandoff
  \ifnum\catcode`\=\string=\active\shorthandoff{=}\fi
  \ifnum\catcode`\"=\active\shorthandoff{"}\fi
\fi

\maketitle
\sphinxtableofcontents
\phantomsection\label{\detokenize{index::doc}}



\section{Quick Start Guide}
\label{\detokenize{quick_start/quick_start:quick-start-guide}}\label{\detokenize{quick_start/quick_start::doc}}

\subsection{Project Introduction}
\label{\detokenize{quick_start/quick_start:project-introduction}}\begin{quote}
\begin{description}
\item[{RFAPI:}] \leavevmode\begin{itemize}
\item {} 
In eagletest/py\_script/hal

\item {} 
it is to store software interface APIs.

\end{itemize}

\item[{RFLIB:}] \leavevmode\begin{itemize}
\item {} 
In eagletest/py\_script/rftest/rflib

\item {} 
it is to store common RF test functions.

\end{itemize}

\item[{TestCase:}] \leavevmode\begin{itemize}
\item {} 
In eagletest/py\_script/rftest/testcase

\item {} 
it is to store RF test cases.

\end{itemize}

\item[{RFDATA:}] \leavevmode\begin{itemize}
\item {} 
In eagletest/py\_script/rftest/rfdata

\item {} 
it is to store RF test data.

\end{itemize}

\item[{DOCS:}] \leavevmode\begin{itemize}
\item {} 
In eagletest/py\_script/rftest/docs

\item {} 
it is to store document and testreport.

\end{itemize}

\end{description}
\end{quote}


\subsection{Start rftest}
\label{\detokenize{quick_start/quick_start:start-rftest}}\begin{quote}

In “eagletest/py\_script/” run commands:
\begin{quote}

ipython -i autotest.py rftest

chip=RFTCS(com(6))   \# 6 should change to your com port number
\end{quote}
\end{quote}


\subsection{Common test commands}
\label{\detokenize{quick_start/quick_start:common-test-commands}}\begin{enumerate}
\def\theenumi{\arabic{enumi}}
\def\labelenumi{\theenumi .}
\makeatletter\def\p@enumii{\p@enumi \theenumi .}\makeatother
\item {} 
write/read memory
\begin{quote}

chip.mem.wrm(0x600060a0, 11, 8, 0x0)

chip.mem.rdm(0x600060a0, 11, 8)
\end{quote}

\item {} 
write/read i2c
\begin{quote}

chip.i2c.rfrx.dtest = 0

chip.i2c.rfrx.dtest
\end{quote}

\item {} 
write/read PBUS
\begin{quote}

chip.pbus.pbus\_debugmode()

chip.pbus.pbus\_wr(‘rfrx1’, ‘en1’, 0x184)

chip.pbus.pbus\_rd(‘rfrx1’, ‘en1’)

chip.pbus.pbus\_workmode()
\end{quote}

\item {} 
select RF channel
\begin{quote}

chip.wifi.rfchsel(1)  \# 1\textasciitilde{}14
\end{quote}

\item {} 
TX Tone
\begin{quote}

\# Debug mode TX tone

chip.pbus.pbus\_debugmode()

chip.pbus.open\_tx(pa=0x5f, bb=0x120)  \# set PA and BB gain

chip.rfcal.tos()    \# calibrate TX DC

chip.wifi.txtone(1, 2, 40)   \#enable, frequency(mHz), digital attenuation(0.24db)

\# Debug mode stop TX tone

chip.wifi.stoptone()   \#stop TX tone

chip.pbus.pbus\_workmode()  \# exit PBUS debug mode

\# Work mode TX tone

chip.wifi.force\_txon(1)

chip.wifi.txtone(1, 2, 40)   \#enable, frequency(mHz), digital attenuation(0.24db)

\# Work mode stop TX tone

chip.wifi.stoptone()

chip.wifi.force\_txon(0)
\end{quote}

\item {} 
RX dump test
\begin{quote}

chip.dump.adcdumptest(dump\_num=1024, plot\_en=1)
\end{quote}

\end{enumerate}


\subsection{Basic functions test cases}
\label{\detokenize{quick_start/quick_start:basic-functions-test-cases}}\begin{enumerate}
\def\theenumi{\arabic{enumi}}
\def\labelenumi{\theenumi .}
\makeatletter\def\p@enumii{\p@enumi \theenumi .}\makeatother
\item {} 
Write and read test
\begin{quote}

basic=BasicTest(chip.comport, chip.chipv)

basic.mem\_test()

basic.i2c\_test()

basic.pbus\_test()
\end{quote}

\item {} 
RFPLL CAP Sweep
\begin{quote}

basic=BasicTest(chip.comport, chip.chipv)

basic.rfpll\_sweep()
\end{quote}

\end{enumerate}


\subsection{Performance oprimization cases}
\label{\detokenize{quick_start/quick_start:performance-oprimization-cases}}

\subsection{TX/RX performance test}
\label{\detokenize{quick_start/quick_start:tx-rx-performance-test}}\begin{enumerate}
\def\theenumi{\arabic{enumi}}
\def\labelenumi{\theenumi .}
\makeatletter\def\p@enumii{\p@enumi \theenumi .}\makeatother
\item {} 
open instrument server
\begin{quote}

instru\_server(‘iqx’)  \#input: ‘iqv’, ‘iqx’, ‘wt’
\end{quote}

\item {} 
TX Power \& EVM \& MASK Test
\begin{quote}

test=WIFI\_TXRX\_TEST(chip.comport, chip.chipv)

test.WIFI\_TX\_PWR\_EVM(cable\_lose=2, channel={[}14{]}, data\_rate={[}‘mcs7’{]}, iqv\_no=2, iqv\_num=10)
\end{quote}

\item {} 
RX Sensitivity Test
\begin{quote}

test.WIFI\_RX\_sens(cable\_lose=2, chan\_in={[}14{]}, data\_rate={[}‘mcs7’{]},iqv\_no=1)
\end{quote}

\item {} 
RX Maximum input level Test
\begin{quote}

test.WIFI\_RX\_maxlevel(cable\_lose=2, chan\_in={[}14{]}, data\_rate={[}‘mcs7’{]}, iqv\_no=1)
\end{quote}

\item {} 
RX Dynamic Range Test
\begin{quote}

test.WIFI\_RX\_range(cable\_lose=2, chan\_in={[}14{]}, data\_rate={[}‘mcs7’{]}, rx\_range={[}‘{[}-75, 0{]}’{]}, iqv\_no=1)
\end{quote}

\end{enumerate}


\subsection{Current test}
\label{\detokenize{quick_start/quick_start:current-test}}

\subsection{Generate docomments}
\label{\detokenize{quick_start/quick_start:generate-docomments}}\begin{quote}

In “eagletest/py\_script/rftest/docs” run commands:
\begin{quote}

ipython doc\_gen.py rftest
\end{quote}
\end{quote}


\section{RFAPI}
\label{\detokenize{rfapi/index:rfapi}}\label{\detokenize{rfapi/index::doc}}

\subsection{common}
\label{\detokenize{rfapi/index:module-common}}\label{\detokenize{rfapi/index:common}}\index{common (模块)}\index{CHIP\_ID (common 中的类)}

\begin{fulllineitems}
\phantomsection\label{\detokenize{rfapi/index:common.CHIP_ID}}\pysiglinewithargsret{\sphinxbfcode{\sphinxupquote{class }}\sphinxcode{\sphinxupquote{common.}}\sphinxbfcode{\sphinxupquote{CHIP\_ID}}}{\emph{channel}, \emph{chipv='AUTO'}}{}
基类:\sphinxcode{\sphinxupquote{object}}

docstring for CHIP\_ID
\index{chip\_mac() (common.CHIP\_ID 方法)}

\begin{fulllineitems}
\phantomsection\label{\detokenize{rfapi/index:common.CHIP_ID.chip_mac}}\pysiglinewithargsret{\sphinxbfcode{\sphinxupquote{chip\_mac}}}{}{}
\end{fulllineitems}


\end{fulllineitems}

\index{CHIP\_INFO (common 中的类)}

\begin{fulllineitems}
\phantomsection\label{\detokenize{rfapi/index:common.CHIP_INFO}}\pysiglinewithargsret{\sphinxbfcode{\sphinxupquote{class }}\sphinxcode{\sphinxupquote{common.}}\sphinxbfcode{\sphinxupquote{CHIP\_INFO}}}{\emph{channel}, \emph{INchipv='AUTO'}}{}
基类:\sphinxcode{\sphinxupquote{object}}

docstring for CHIP\_INFO
\index{get\_chipv() (common.CHIP\_INFO 方法)}

\begin{fulllineitems}
\phantomsection\label{\detokenize{rfapi/index:common.CHIP_INFO.get_chipv}}\pysiglinewithargsret{\sphinxbfcode{\sphinxupquote{get\_chipv}}}{}{}
Use UART Date register to determine the chip version

\end{fulllineitems}


\end{fulllineitems}

\index{I2C (common 中的类)}

\begin{fulllineitems}
\phantomsection\label{\detokenize{rfapi/index:common.I2C}}\pysiglinewithargsret{\sphinxbfcode{\sphinxupquote{class }}\sphinxcode{\sphinxupquote{common.}}\sphinxbfcode{\sphinxupquote{I2C}}}{\emph{channel}, \emph{chipv='ESP32'}}{}
基类:\sphinxcode{\sphinxupquote{object}}

docstring for I2C
\index{i2c\_rd() (common.I2C 方法)}

\begin{fulllineitems}
\phantomsection\label{\detokenize{rfapi/index:common.I2C.i2c_rd}}\pysiglinewithargsret{\sphinxbfcode{\sphinxupquote{i2c\_rd}}}{\emph{block}, \emph{host\_id}, \emph{reg\_addr}}{}
\end{fulllineitems}

\index{i2c\_rdm() (common.I2C 方法)}

\begin{fulllineitems}
\phantomsection\label{\detokenize{rfapi/index:common.I2C.i2c_rdm}}\pysiglinewithargsret{\sphinxbfcode{\sphinxupquote{i2c\_rdm}}}{\emph{block}, \emph{host\_id}, \emph{reg\_addr}, \emph{msb}, \emph{lsb}}{}
\end{fulllineitems}

\index{i2c\_wr() (common.I2C 方法)}

\begin{fulllineitems}
\phantomsection\label{\detokenize{rfapi/index:common.I2C.i2c_wr}}\pysiglinewithargsret{\sphinxbfcode{\sphinxupquote{i2c\_wr}}}{\emph{block}, \emph{host\_id}, \emph{reg\_addr}, \emph{value}}{}
\end{fulllineitems}

\index{i2c\_wrm() (common.I2C 方法)}

\begin{fulllineitems}
\phantomsection\label{\detokenize{rfapi/index:common.I2C.i2c_wrm}}\pysiglinewithargsret{\sphinxbfcode{\sphinxupquote{i2c\_wrm}}}{\emph{block}, \emph{host\_id}, \emph{reg\_addr}, \emph{msb}, \emph{lsb}, \emph{value}}{}
\end{fulllineitems}


\end{fulllineitems}

\index{MEM (common 中的类)}

\begin{fulllineitems}
\phantomsection\label{\detokenize{rfapi/index:common.MEM}}\pysiglinewithargsret{\sphinxbfcode{\sphinxupquote{class }}\sphinxcode{\sphinxupquote{common.}}\sphinxbfcode{\sphinxupquote{MEM}}}{\emph{channel}, \emph{chipv='ESP32'}}{}
基类:\sphinxcode{\sphinxupquote{object}}

docstring for common
\index{accumiq() (common.MEM 方法)}

\begin{fulllineitems}
\phantomsection\label{\detokenize{rfapi/index:common.MEM.accumiq}}\pysiglinewithargsret{\sphinxbfcode{\sphinxupquote{accumiq}}}{\emph{mem\_addr}, \emph{burst\_len}}{}
\end{fulllineitems}

\index{clrmask() (common.MEM 方法)}

\begin{fulllineitems}
\phantomsection\label{\detokenize{rfapi/index:common.MEM.clrmask}}\pysiglinewithargsret{\sphinxbfcode{\sphinxupquote{clrmask}}}{\emph{reg\_addr}, \emph{mask}}{}
\end{fulllineitems}

\index{rd() (common.MEM 方法)}

\begin{fulllineitems}
\phantomsection\label{\detokenize{rfapi/index:common.MEM.rd}}\pysiglinewithargsret{\sphinxbfcode{\sphinxupquote{rd}}}{\emph{reg\_addr}}{}
\end{fulllineitems}

\index{rdm() (common.MEM 方法)}

\begin{fulllineitems}
\phantomsection\label{\detokenize{rfapi/index:common.MEM.rdm}}\pysiglinewithargsret{\sphinxbfcode{\sphinxupquote{rdm}}}{\emph{reg\_addr}, \emph{msb}, \emph{lsb}}{}
\end{fulllineitems}

\index{rdmem() (common.MEM 方法)}

\begin{fulllineitems}
\phantomsection\label{\detokenize{rfapi/index:common.MEM.rdmem}}\pysiglinewithargsret{\sphinxbfcode{\sphinxupquote{rdmem}}}{\emph{mem\_addr}, \emph{data\_len}}{}
\end{fulllineitems}

\index{setmask() (common.MEM 方法)}

\begin{fulllineitems}
\phantomsection\label{\detokenize{rfapi/index:common.MEM.setmask}}\pysiglinewithargsret{\sphinxbfcode{\sphinxupquote{setmask}}}{\emph{reg\_addr}, \emph{mask}}{}
\end{fulllineitems}

\index{wr() (common.MEM 方法)}

\begin{fulllineitems}
\phantomsection\label{\detokenize{rfapi/index:common.MEM.wr}}\pysiglinewithargsret{\sphinxbfcode{\sphinxupquote{wr}}}{\emph{reg\_addr}, \emph{value}}{}
\end{fulllineitems}

\index{wrm() (common.MEM 方法)}

\begin{fulllineitems}
\phantomsection\label{\detokenize{rfapi/index:common.MEM.wrm}}\pysiglinewithargsret{\sphinxbfcode{\sphinxupquote{wrm}}}{\emph{reg\_addr}, \emph{msb}, \emph{lsb}, \emph{value}}{}
\end{fulllineitems}


\end{fulllineitems}

\index{PBUS (common 中的类)}

\begin{fulllineitems}
\phantomsection\label{\detokenize{rfapi/index:common.PBUS}}\pysiglinewithargsret{\sphinxbfcode{\sphinxupquote{class }}\sphinxcode{\sphinxupquote{common.}}\sphinxbfcode{\sphinxupquote{PBUS}}}{\emph{channel}, \emph{chipv='ESP8266'}}{}
基类:\sphinxcode{\sphinxupquote{object}}

docstring for PBUS
\index{pbus\_debugmode() (common.PBUS 方法)}

\begin{fulllineitems}
\phantomsection\label{\detokenize{rfapi/index:common.PBUS.pbus_debugmode}}\pysiglinewithargsret{\sphinxbfcode{\sphinxupquote{pbus\_debugmode}}}{}{}
\end{fulllineitems}

\index{pbus\_rd() (common.PBUS 方法)}

\begin{fulllineitems}
\phantomsection\label{\detokenize{rfapi/index:common.PBUS.pbus_rd}}\pysiglinewithargsret{\sphinxbfcode{\sphinxupquote{pbus\_rd}}}{\emph{pbus\_sel}, \emph{pbus\_en\_sel}}{}
\end{fulllineitems}

\index{pbus\_rm() (common.PBUS 方法)}

\begin{fulllineitems}
\phantomsection\label{\detokenize{rfapi/index:common.PBUS.pbus_rm}}\pysiglinewithargsret{\sphinxbfcode{\sphinxupquote{pbus\_rm}}}{\emph{pbus\_sel}, \emph{pbus\_en\_sel}, \emph{msb}, \emph{lsb}}{}
\end{fulllineitems}

\index{pbus\_wm() (common.PBUS 方法)}

\begin{fulllineitems}
\phantomsection\label{\detokenize{rfapi/index:common.PBUS.pbus_wm}}\pysiglinewithargsret{\sphinxbfcode{\sphinxupquote{pbus\_wm}}}{\emph{pbus\_sel}, \emph{pbus\_en\_sel}, \emph{msb}, \emph{lsb}, \emph{value}}{}
\end{fulllineitems}

\index{pbus\_workmode() (common.PBUS 方法)}

\begin{fulllineitems}
\phantomsection\label{\detokenize{rfapi/index:common.PBUS.pbus_workmode}}\pysiglinewithargsret{\sphinxbfcode{\sphinxupquote{pbus\_workmode}}}{}{}
\end{fulllineitems}

\index{pbus\_wr() (common.PBUS 方法)}

\begin{fulllineitems}
\phantomsection\label{\detokenize{rfapi/index:common.PBUS.pbus_wr}}\pysiglinewithargsret{\sphinxbfcode{\sphinxupquote{pbus\_wr}}}{\emph{pbus\_sel}, \emph{pbus\_en\_sel}, \emph{value}}{}
\end{fulllineitems}


\end{fulllineitems}



\subsection{wifi\_api}
\label{\detokenize{rfapi/index:module-wifi_api}}\label{\detokenize{rfapi/index:wifi-api}}\index{wifi\_api (模块)}\index{WIFIAPI (wifi\_api 中的类)}

\begin{fulllineitems}
\phantomsection\label{\detokenize{rfapi/index:wifi_api.WIFIAPI}}\pysiglinewithargsret{\sphinxbfcode{\sphinxupquote{class }}\sphinxcode{\sphinxupquote{wifi\_api.}}\sphinxbfcode{\sphinxupquote{WIFIAPI}}}{\emph{channel}, \emph{chipv='ESP32'}}{}
基类:\sphinxcode{\sphinxupquote{object}}
\index{adctrig() (wifi\_api.WIFIAPI 方法)}

\begin{fulllineitems}
\phantomsection\label{\detokenize{rfapi/index:wifi_api.WIFIAPI.adctrig}}\pysiglinewithargsret{\sphinxbfcode{\sphinxupquote{adctrig}}}{\emph{smp\_num\_aft\_trig}, \emph{trig\_mode='sw'}, \emph{sample\_80m=0}, \emph{trigcase=0}, \emph{dump\_trig=0}, \emph{rx\_gain\_mode=0}, \emph{rx\_gain=0}, \emph{rx\_gain0=0}, \emph{gain0\_wait=0}}{}
return curr\_ptr,wrap\_flag, buff\_addr, buff\_size

\end{fulllineitems}

\index{bbinit() (wifi\_api.WIFIAPI 方法)}

\begin{fulllineitems}
\phantomsection\label{\detokenize{rfapi/index:wifi_api.WIFIAPI.bbinit}}\pysiglinewithargsret{\sphinxbfcode{\sphinxupquote{bbinit}}}{}{}~\begin{quote}\begin{description}
\item[{Brief}] \leavevmode
bb init

\item[{Param}] \leavevmode\begin{itemize}
\item {} 
no param

\end{itemize}

\item[{返回}] \leavevmode
\begin{itemize}
\item {} 
no return

\end{itemize}


\end{description}\end{quote}

\end{fulllineitems}

\index{cbw40m\_en() (wifi\_api.WIFIAPI 方法)}

\begin{fulllineitems}
\phantomsection\label{\detokenize{rfapi/index:wifi_api.WIFIAPI.cbw40m_en}}\pysiglinewithargsret{\sphinxbfcode{\sphinxupquote{cbw40m\_en}}}{\emph{en=0}}{}~\begin{quote}\begin{description}
\item[{Brief}] \leavevmode
40M bandwidth enable

\item[{Param}] \leavevmode\begin{itemize}
\item {} 
en:

\item {} 
0: HT40 disable

\item {} 
1: HT40 enable

\end{itemize}

\item[{返回}] \leavevmode
\begin{itemize}
\item {} 
print the status

\end{itemize}


\end{description}\end{quote}

\end{fulllineitems}

\index{cmdstop() (wifi\_api.WIFIAPI 方法)}

\begin{fulllineitems}
\phantomsection\label{\detokenize{rfapi/index:wifi_api.WIFIAPI.cmdstop}}\pysiglinewithargsret{\sphinxbfcode{\sphinxupquote{cmdstop}}}{}{}~\begin{quote}\begin{description}
\item[{Brief}] \leavevmode
wifi TX/RX state stop

\item[{Param}] \leavevmode\begin{itemize}
\item {} 
no param

\end{itemize}

\item[{返回}] \leavevmode
\begin{itemize}
\item {} 
no return

\end{itemize}


\end{description}\end{quote}

\end{fulllineitems}

\index{esp\_origin\_mac() (wifi\_api.WIFIAPI 方法)}

\begin{fulllineitems}
\phantomsection\label{\detokenize{rfapi/index:wifi_api.WIFIAPI.esp_origin_mac}}\pysiglinewithargsret{\sphinxbfcode{\sphinxupquote{esp\_origin\_mac}}}{}{}~\begin{quote}\begin{description}
\item[{Brief}] \leavevmode
origin mac read

\item[{Param}] \leavevmode\begin{itemize}
\item {} 
no param

\end{itemize}

\item[{返回}] \leavevmode
\begin{itemize}
\item {} 
the value of origin mac

\end{itemize}


\end{description}\end{quote}

\end{fulllineitems}

\index{esp\_rx() (wifi\_api.WIFIAPI 方法)}

\begin{fulllineitems}
\phantomsection\label{\detokenize{rfapi/index:wifi_api.WIFIAPI.esp_rx}}\pysiglinewithargsret{\sphinxbfcode{\sphinxupquote{esp\_rx}}}{\emph{chan=1}, \emph{data\_rate=23}}{}~\begin{quote}\begin{description}
\item[{Brief}] \leavevmode
rx packect command

\item[{Param}] \leavevmode\begin{itemize}
\item {} 
chan: channel number (1 to 14)

\item {} \begin{description}
\item[{data\_rate:}] \leavevmode

\begin{savenotes}\sphinxattablestart
\centering
\begin{tabulary}{\linewidth}[t]{|T|T|T|T|T|T|}
\hline
\sphinxstyletheadfamily 
11b
&\sphinxstyletheadfamily &\sphinxstyletheadfamily 
11g
&\sphinxstyletheadfamily &\sphinxstyletheadfamily 
11n
&\sphinxstyletheadfamily \\
\hline\sphinxstyletheadfamily 
param
&\sphinxstyletheadfamily 
rate
&\sphinxstyletheadfamily 
param
&\sphinxstyletheadfamily 
rate
&\sphinxstyletheadfamily 
param
&\sphinxstyletheadfamily 
rate
\\
\hline
0x0
&
1M
&
0xb
&
6M
&
0x10
&
MCS0
\\
\hline
0x1
&
2ML
&
0xf
&
9M
&
0x11
&
MCS1
\\
\hline
0x2
&
5.5ML
&
0xa
&
12M
&
0x12
&
MCS2
\\
\hline
0x3
&
11ML
&
0xe
&
18M
&
0x13
&
MCS3
\\
\hline
0x4
&
\textendash{}
&
0x9
&
24M
&
0x14
&
MCS4
\\
\hline
0x5
&
2MS
&
0xd
&
36M
&
0x15
&
MCS5
\\
\hline
0x6
&
5.5MS
&
0x8
&
48M
&
0x16
&
MCS6
\\
\hline
0x7
&
11MS
&
0xc
&
54M
&
0x17
&
MCS7
\\
\hline
\end{tabulary}
\par
\sphinxattableend\end{savenotes}

\end{description}

\end{itemize}

\item[{返回}] \leavevmode
\begin{itemize}
\item {} 
print the status

\end{itemize}


\end{description}\end{quote}

\end{fulllineitems}

\index{esp\_tx() (wifi\_api.WIFIAPI 方法)}

\begin{fulllineitems}
\phantomsection\label{\detokenize{rfapi/index:wifi_api.WIFIAPI.esp_tx}}\pysiglinewithargsret{\sphinxbfcode{\sphinxupquote{esp\_tx}}}{\emph{chan=1}, \emph{data\_rate=23}, \emph{backoff=0}}{}~\begin{quote}\begin{description}
\item[{Brief}] \leavevmode
tx packect command

\item[{Param}] \leavevmode\begin{itemize}
\item {} 
chan: channel number (1 to 14)

\item {} \begin{description}
\item[{data\_rate:}] \leavevmode

\begin{savenotes}\sphinxattablestart
\centering
\begin{tabulary}{\linewidth}[t]{|T|T|T|T|T|T|}
\hline
\sphinxstyletheadfamily 
11b
&\sphinxstyletheadfamily &\sphinxstyletheadfamily 
11g
&\sphinxstyletheadfamily &\sphinxstyletheadfamily 
11n
&\sphinxstyletheadfamily \\
\hline\sphinxstyletheadfamily 
param
&\sphinxstyletheadfamily 
rate
&\sphinxstyletheadfamily 
param
&\sphinxstyletheadfamily 
rate
&\sphinxstyletheadfamily 
param
&\sphinxstyletheadfamily 
rate
\\
\hline
0x0
&
1M
&
0xb
&
6M
&
0x10
&
MCS0
\\
\hline
0x1
&
2ML
&
0xf
&
9M
&
0x11
&
MCS1
\\
\hline
0x2
&
5.5ML
&
0xa
&
12M
&
0x12
&
MCS2
\\
\hline
0x3
&
11ML
&
0xe
&
18M
&
0x13
&
MCS3
\\
\hline
0x4
&
\textendash{}
&
0x9
&
24M
&
0x14
&
MCS4
\\
\hline
0x5
&
2MS
&
0xd
&
36M
&
0x15
&
MCS5
\\
\hline
0x6
&
5.5MS
&
0x8
&
48M
&
0x16
&
MCS6
\\
\hline
0x7
&
11MS
&
0xc
&
54M
&
0x17
&
MCS7
\\
\hline
\end{tabulary}
\par
\sphinxattableend\end{savenotes}

\end{description}

\item {} 
backoff: tx power attenuation, 4 indicates an attenuation 1dB

\end{itemize}

\item[{返回}] \leavevmode
\begin{itemize}
\item {} 
print the status

\end{itemize}


\end{description}\end{quote}

\end{fulllineitems}

\index{filltxpacket() (wifi\_api.WIFIAPI 方法)}

\begin{fulllineitems}
\phantomsection\label{\detokenize{rfapi/index:wifi_api.WIFIAPI.filltxpacket}}\pysiglinewithargsret{\sphinxbfcode{\sphinxupquote{filltxpacket}}}{\emph{PackLen}, \emph{pdu0len}, \emph{pdu1len}, \emph{rate=0}, \emph{key\_no=0}, \emph{bssid\_no=0}, \emph{lnkstartaddr=0}, \emph{gi\_type='long'}, \emph{ap\_mac\_5=1}, \emph{ap\_mac\_4=2}, \emph{ap\_mac\_3=3}, \emph{ap\_mac\_2=4}, \emph{ap\_mac\_1=5}, \emph{ap\_mac\_0=6}}{}~\begin{quote}\begin{description}
\item[{Brief}] \leavevmode
wifi TX tone set

\item[{Param}] \leavevmode\begin{itemize}
\item {} 
PackLen: include que\_no and packeet\_len,as (que\_no\textless{}\textless{}16)+packlen

\item {} 
pdu0len: include pdu0 and pdu2 length, as (pdu2len\textless{}\textless{}16)+pdu0len

\item {} 
pdu1len: include pdu1 and pdu3 length, as (pdu3len\textless{}\textless{}16)+pdu1len

\item {} 
for ampsdu and ampdu, pdu0len and pdu1len must set to zero

\item {} 
lnkstartaddr:for ver5.0 above, it \textless{}\textless{}8+bssid\_no as param to board

\end{itemize}

\end{description}\end{quote}

\end{fulllineitems}

\index{get\_rx\_tone\_pwr() (wifi\_api.WIFIAPI 方法)}

\begin{fulllineitems}
\phantomsection\label{\detokenize{rfapi/index:wifi_api.WIFIAPI.get_rx_tone_pwr}}\pysiglinewithargsret{\sphinxbfcode{\sphinxupquote{get\_rx\_tone\_pwr}}}{\emph{rx\_freq\_cfg}}{}
\end{fulllineitems}

\index{init\_print() (wifi\_api.WIFIAPI 方法)}

\begin{fulllineitems}
\phantomsection\label{\detokenize{rfapi/index:wifi_api.WIFIAPI.init_print}}\pysiglinewithargsret{\sphinxbfcode{\sphinxupquote{init\_print}}}{}{}~\begin{quote}\begin{description}
\item[{Brief}] \leavevmode
init print

\item[{Param}] \leavevmode\begin{itemize}
\item {} 
no param

\end{itemize}

\item[{返回}] \leavevmode
\begin{itemize}
\item {} 
no return

\end{itemize}


\end{description}\end{quote}

\end{fulllineitems}

\index{macinit() (wifi\_api.WIFIAPI 方法)}

\begin{fulllineitems}
\phantomsection\label{\detokenize{rfapi/index:wifi_api.WIFIAPI.macinit}}\pysiglinewithargsret{\sphinxbfcode{\sphinxupquote{macinit}}}{}{}~\begin{quote}\begin{description}
\item[{Brief}] \leavevmode
mac init

\item[{Param}] \leavevmode\begin{itemize}
\item {} 
no param

\end{itemize}

\item[{返回}] \leavevmode
\begin{itemize}
\item {} 
no return

\end{itemize}


\end{description}\end{quote}

\end{fulllineitems}

\index{phyinit() (wifi\_api.WIFIAPI 方法)}

\begin{fulllineitems}
\phantomsection\label{\detokenize{rfapi/index:wifi_api.WIFIAPI.phyinit}}\pysiglinewithargsret{\sphinxbfcode{\sphinxupquote{phyinit}}}{}{}~\begin{quote}\begin{description}
\item[{Brief}] \leavevmode
phy init(include rfinit and bbinit and macinit)

\item[{Param}] \leavevmode\begin{itemize}
\item {} 
no param

\end{itemize}

\item[{返回}] \leavevmode
\begin{itemize}
\item {} 
no return

\end{itemize}


\end{description}\end{quote}

\end{fulllineitems}

\index{read\_hw\_noisefloor() (wifi\_api.WIFIAPI 方法)}

\begin{fulllineitems}
\phantomsection\label{\detokenize{rfapi/index:wifi_api.WIFIAPI.read_hw_noisefloor}}\pysiglinewithargsret{\sphinxbfcode{\sphinxupquote{read\_hw\_noisefloor}}}{}{}~\begin{quote}\begin{description}
\item[{Brief}] \leavevmode
RF noisefloor read

\item[{Param}] \leavevmode\begin{itemize}
\item {} 
no param

\end{itemize}

\item[{返回}] \leavevmode
\begin{itemize}
\item {} 
the value of hardware noisefloor

\end{itemize}


\end{description}\end{quote}

\end{fulllineitems}

\index{rfchsel() (wifi\_api.WIFIAPI 方法)}

\begin{fulllineitems}
\phantomsection\label{\detokenize{rfapi/index:wifi_api.WIFIAPI.rfchsel}}\pysiglinewithargsret{\sphinxbfcode{\sphinxupquote{rfchsel}}}{\emph{chan}, \emph{cbw2040\_cfg=0}}{}~\begin{quote}\begin{description}
\item[{Brief}] \leavevmode
wifi channel set

\item[{Param}] \leavevmode\begin{itemize}
\item {} 
chan: channel number (1 to 14)

\item {} \begin{description}
\item[{cbw2040\_cfg:}] \leavevmode\begin{itemize}
\item {} 
1:HT40 enable

\item {} 
0:HT40 disable

\end{itemize}

\end{description}

\end{itemize}

\item[{返回}] \leavevmode
\begin{itemize}
\item {} 
no return

\end{itemize}


\end{description}\end{quote}

\end{fulllineitems}

\index{rfinit() (wifi\_api.WIFIAPI 方法)}

\begin{fulllineitems}
\phantomsection\label{\detokenize{rfapi/index:wifi_api.WIFIAPI.rfinit}}\pysiglinewithargsret{\sphinxbfcode{\sphinxupquote{rfinit}}}{}{}~\begin{quote}\begin{description}
\item[{Brief}] \leavevmode
rf init

\item[{Param}] \leavevmode\begin{itemize}
\item {} 
no param

\end{itemize}

\item[{返回}] \leavevmode
\begin{itemize}
\item {} 
no return

\end{itemize}


\end{description}\end{quote}

\end{fulllineitems}

\index{rxdc\_cal() (wifi\_api.WIFIAPI 方法)}

\begin{fulllineitems}
\phantomsection\label{\detokenize{rfapi/index:wifi_api.WIFIAPI.rxdc_cal}}\pysiglinewithargsret{\sphinxbfcode{\sphinxupquote{rxdc\_cal}}}{}{}
\end{fulllineitems}

\index{rxstart() (wifi\_api.WIFIAPI 方法)}

\begin{fulllineitems}
\phantomsection\label{\detokenize{rfapi/index:wifi_api.WIFIAPI.rxstart}}\pysiglinewithargsret{\sphinxbfcode{\sphinxupquote{rxstart}}}{\emph{rate\_sym}}{}~\begin{quote}\begin{description}
\item[{Brief}] \leavevmode
wifi RX state open

\item[{Param}] \leavevmode\begin{itemize}
\item {} 
rate\_sym:  wifi rate (need to measure RX performance)

\end{itemize}

\item[{返回}] \leavevmode
\begin{itemize}
\item {} 
no return

\end{itemize}


\end{description}\end{quote}

\end{fulllineitems}

\index{set\_noise\_floor() (wifi\_api.WIFIAPI 方法)}

\begin{fulllineitems}
\phantomsection\label{\detokenize{rfapi/index:wifi_api.WIFIAPI.set_noise_floor}}\pysiglinewithargsret{\sphinxbfcode{\sphinxupquote{set\_noise\_floor}}}{\emph{noise=380}}{}~\begin{quote}\begin{description}
\item[{Brief}] \leavevmode
RF noisefloor set

\item[{Param}] \leavevmode\begin{itemize}
\item {} 
noise:  value of noisefloor

\end{itemize}

\item[{返回}] \leavevmode
\begin{itemize}
\item {} 
no return

\end{itemize}


\end{description}\end{quote}

\end{fulllineitems}

\index{set\_tx\_dig\_gain() (wifi\_api.WIFIAPI 方法)}

\begin{fulllineitems}
\phantomsection\label{\detokenize{rfapi/index:wifi_api.WIFIAPI.set_tx_dig_gain}}\pysiglinewithargsret{\sphinxbfcode{\sphinxupquote{set\_tx\_dig\_gain}}}{\emph{force\_en=1}, \emph{dig\_gain=25}}{}~\begin{quote}\begin{description}
\item[{Brief}] \leavevmode
set tx digital gain command

\item[{Param}] \leavevmode\begin{itemize}
\item {} \begin{description}
\item[{force\_en:}] \leavevmode\begin{itemize}
\item {} 
0: disable;

\item {} 
1: enable

\end{itemize}

\end{description}

\item {} 
dig\_gain:

\end{itemize}

\item[{返回}] \leavevmode
\begin{itemize}
\item {} 
print the status

\end{itemize}


\end{description}\end{quote}

\end{fulllineitems}

\index{set\_tx\_gain() (wifi\_api.WIFIAPI 方法)}

\begin{fulllineitems}
\phantomsection\label{\detokenize{rfapi/index:wifi_api.WIFIAPI.set_tx_gain}}\pysiglinewithargsret{\sphinxbfcode{\sphinxupquote{set\_tx\_gain}}}{\emph{pa\_gain=95}, \emph{bb\_gain=288}}{}~\begin{quote}\begin{description}
\item[{Brief}] \leavevmode
set tx gain command

\item[{Param}] \leavevmode\begin{itemize}
\item {} 
pa\_gain: 0x1f, 0x2f,0x3f,0x4f,0x5f,0x6f,0x7f

\item {} 
bb\_gain: …, 0x100,0x140,0x20,0x60, …

\end{itemize}

\item[{返回}] \leavevmode
\begin{itemize}
\item {} 
print the status

\end{itemize}


\end{description}\end{quote}

\end{fulllineitems}

\index{stoptone() (wifi\_api.WIFIAPI 方法)}

\begin{fulllineitems}
\phantomsection\label{\detokenize{rfapi/index:wifi_api.WIFIAPI.stoptone}}\pysiglinewithargsret{\sphinxbfcode{\sphinxupquote{stoptone}}}{\emph{tone\_no=0}}{}~\begin{quote}\begin{description}
\item[{Brief}] \leavevmode
wifi tx tone state close

\item[{Param}] \leavevmode\begin{itemize}
\item {} 
tone\_no:  the number of tone

\end{itemize}

\item[{返回}] \leavevmode
\begin{itemize}
\item {} 
no return

\end{itemize}


\end{description}\end{quote}

\end{fulllineitems}

\index{target\_power\_backoff() (wifi\_api.WIFIAPI 方法)}

\begin{fulllineitems}
\phantomsection\label{\detokenize{rfapi/index:wifi_api.WIFIAPI.target_power_backoff}}\pysiglinewithargsret{\sphinxbfcode{\sphinxupquote{target\_power\_backoff}}}{\emph{backoff\_qdb}}{}~\begin{quote}\begin{description}
\item[{Brief}] \leavevmode
wifi target power backoff

\item[{Param}] \leavevmode\begin{itemize}
\item {} 
backoff\_qdb: value of backoff

\end{itemize}

\item[{返回}] \leavevmode
\begin{itemize}
\item {} 
no return

\end{itemize}


\end{description}\end{quote}

\end{fulllineitems}

\index{test\_txtone\_pwr() (wifi\_api.WIFIAPI 方法)}

\begin{fulllineitems}
\phantomsection\label{\detokenize{rfapi/index:wifi_api.WIFIAPI.test_txtone_pwr}}\pysiglinewithargsret{\sphinxbfcode{\sphinxupquote{test\_txtone\_pwr}}}{\emph{atten}, \emph{loop\_num}, \emph{mode=0}, \emph{step=0}, \emph{delay\_us=10}}{}
\end{fulllineitems}

\index{tx\_cbw40m\_en() (wifi\_api.WIFIAPI 方法)}

\begin{fulllineitems}
\phantomsection\label{\detokenize{rfapi/index:wifi_api.WIFIAPI.tx_cbw40m_en}}\pysiglinewithargsret{\sphinxbfcode{\sphinxupquote{tx\_cbw40m\_en}}}{\emph{en=0}}{}~\begin{quote}\begin{description}
\item[{Brief}] \leavevmode
Tx CBW40 enable

\item[{Param}] \leavevmode\begin{itemize}
\item {} 
en:

\item {} 
0: Tx CBW40 disable

\item {} 
1: Tx CBW40 enable

\end{itemize}

\item[{返回}] \leavevmode
\begin{itemize}
\item {} 
print the status

\end{itemize}


\end{description}\end{quote}

\end{fulllineitems}

\index{tx\_contin\_en() (wifi\_api.WIFIAPI 方法)}

\begin{fulllineitems}
\phantomsection\label{\detokenize{rfapi/index:wifi_api.WIFIAPI.tx_contin_en}}\pysiglinewithargsret{\sphinxbfcode{\sphinxupquote{tx\_contin\_en}}}{\emph{en=0}}{}~\begin{quote}\begin{description}
\item[{Brief}] \leavevmode
Tx continuous enable

\item[{Param}] \leavevmode\begin{itemize}
\item {} 
en:

\item {} 
0: Tx continuous disable

\item {} 
1: Tx continuous enable

\end{itemize}

\item[{返回}] \leavevmode
\begin{itemize}
\item {} 
print the status

\end{itemize}


\end{description}\end{quote}

\end{fulllineitems}

\index{txstart() (wifi\_api.WIFIAPI 方法)}

\begin{fulllineitems}
\phantomsection\label{\detokenize{rfapi/index:wifi_api.WIFIAPI.txstart}}\pysiglinewithargsret{\sphinxbfcode{\sphinxupquote{txstart}}}{\emph{tx\_rate}, \emph{packnum}, \emph{que\_no=1}, \emph{frm\_delay=100}, \emph{cbw40=0}, \emph{ht\_dup=0}, \emph{dis\_cca=1}}{}
\end{fulllineitems}

\index{txtone() (wifi\_api.WIFIAPI 方法)}

\begin{fulllineitems}
\phantomsection\label{\detokenize{rfapi/index:wifi_api.WIFIAPI.txtone}}\pysiglinewithargsret{\sphinxbfcode{\sphinxupquote{txtone}}}{\emph{tone1\_en=1}, \emph{freq1\_mhz=2}, \emph{tone1\_att=0}, \emph{tone2\_en=0}, \emph{freq2\_mhz=0}, \emph{tone2\_att=0}}{}~\begin{quote}\begin{description}
\item[{Brief}] \leavevmode
wifi TX tone set

\item[{Param}] \leavevmode\begin{itemize}
\item {} \begin{description}
\item[{tone1\_en:}] \leavevmode\begin{itemize}
\item {} 
1: first tone enable

\item {} 
0: fisrt tone disable

\end{itemize}

\end{description}

\item {} 
freq1\_mhz: first tone offset frequency

\item {} 
tone1\_att: first tone attenuation set

\item {} \begin{description}
\item[{tone2\_en:}] \leavevmode\begin{itemize}
\item {} 
1: second tone enable

\item {} 
0: second tone disable

\end{itemize}

\end{description}

\item {} 
freq2\_mhz: second tone offset frequency

\item {} 
tone2\_att: second tone attenuation set

\end{itemize}

\item[{返回}] \leavevmode
\begin{itemize}
\item {} 
no return

\end{itemize}


\end{description}\end{quote}

\end{fulllineitems}

\index{txtone\_step() (wifi\_api.WIFIAPI 方法)}

\begin{fulllineitems}
\phantomsection\label{\detokenize{rfapi/index:wifi_api.WIFIAPI.txtone_step}}\pysiglinewithargsret{\sphinxbfcode{\sphinxupquote{txtone\_step}}}{\emph{step1=0}, \emph{att1=0}, \emph{en2=0}, \emph{step2=0}, \emph{att2=0}}{}
\end{fulllineitems}

\index{wifiscwout() (wifi\_api.WIFIAPI 方法)}

\begin{fulllineitems}
\phantomsection\label{\detokenize{rfapi/index:wifi_api.WIFIAPI.wifiscwout}}\pysiglinewithargsret{\sphinxbfcode{\sphinxupquote{wifiscwout}}}{\emph{en=1}, \emph{chan=1}, \emph{backoff=0}}{}~\begin{quote}\begin{description}
\item[{Brief}] \leavevmode
SCW TX command

\item[{Param}] \leavevmode\begin{itemize}
\item {} \begin{description}
\item[{en: SCW Tx enable}] \leavevmode\begin{itemize}
\item {} 
0: disable

\item {} 
1: enable

\end{itemize}

\end{description}

\item {} 
chan: channel number (1 to 14)

\item {} 
backoff: tx power attenuation, 4 indicates an attenuation 1dB

\end{itemize}

\item[{返回}] \leavevmode
\begin{itemize}
\item {} 
print the status

\end{itemize}


\end{description}\end{quote}

\end{fulllineitems}

\index{wifitxout() (wifi\_api.WIFIAPI 方法)}

\begin{fulllineitems}
\phantomsection\label{\detokenize{rfapi/index:wifi_api.WIFIAPI.wifitxout}}\pysiglinewithargsret{\sphinxbfcode{\sphinxupquote{wifitxout}}}{\emph{chan=1}, \emph{data\_rate=23}, \emph{backoff=0}}{}~\begin{quote}\begin{description}
\item[{Brief}] \leavevmode
tx packect command

\item[{Param}] \leavevmode\begin{itemize}
\item {} 
chan: channel number (1 to 14)

\item {} \begin{description}
\item[{data\_rate:}] \leavevmode

\begin{savenotes}\sphinxattablestart
\centering
\begin{tabulary}{\linewidth}[t]{|T|T|T|T|T|T|}
\hline
\sphinxstyletheadfamily 
11b
&\sphinxstyletheadfamily &\sphinxstyletheadfamily 
11g
&\sphinxstyletheadfamily &\sphinxstyletheadfamily 
11n
&\sphinxstyletheadfamily \\
\hline\sphinxstyletheadfamily 
param
&\sphinxstyletheadfamily 
rate
&\sphinxstyletheadfamily 
param
&\sphinxstyletheadfamily 
rate
&\sphinxstyletheadfamily 
param
&\sphinxstyletheadfamily 
rate
\\
\hline
0x0
&
1M
&
0xb
&
6M
&
0x10
&
MCS0
\\
\hline
0x1
&
2ML
&
0xf
&
9M
&
0x11
&
MCS1
\\
\hline
0x2
&
5.5ML
&
0xa
&
12M
&
0x12
&
MCS2
\\
\hline
0x3
&
11ML
&
0xe
&
18M
&
0x13
&
MCS3
\\
\hline
0x4
&
\textendash{}
&
0x9
&
24M
&
0x14
&
MCS4
\\
\hline
0x5
&
2MS
&
0xd
&
36M
&
0x15
&
MCS5
\\
\hline
0x6
&
5.5MS
&
0x8
&
48M
&
0x16
&
MCS6
\\
\hline
0x7
&
11MS
&
0xc
&
54M
&
0x17
&
MCS7
\\
\hline
\end{tabulary}
\par
\sphinxattableend\end{savenotes}

\end{description}

\item {} 
backoff: tx power attenuation, 4 indicates an attenuation 1dB

\end{itemize}

\item[{返回}] \leavevmode
\begin{itemize}
\item {} 
print the status

\end{itemize}


\end{description}\end{quote}

\end{fulllineitems}


\end{fulllineitems}



\subsection{bt\_api}
\label{\detokenize{rfapi/index:module-bt_api}}\label{\detokenize{rfapi/index:bt-api}}\index{bt\_api (模块)}\index{BTAPI (bt\_api 中的类)}

\begin{fulllineitems}
\phantomsection\label{\detokenize{rfapi/index:bt_api.BTAPI}}\pysiglinewithargsret{\sphinxbfcode{\sphinxupquote{class }}\sphinxcode{\sphinxupquote{bt\_api.}}\sphinxbfcode{\sphinxupquote{BTAPI}}}{\emph{channel}, \emph{chipv='ESP32'}}{}
基类:\sphinxcode{\sphinxupquote{object}}
\index{bt\_tx\_tone() (bt\_api.BTAPI 方法)}

\begin{fulllineitems}
\phantomsection\label{\detokenize{rfapi/index:bt_api.BTAPI.bt_tx_tone}}\pysiglinewithargsret{\sphinxbfcode{\sphinxupquote{bt\_tx\_tone}}}{\emph{en=0}, \emph{chan=1}, \emph{backoff=0}}{}~\begin{quote}\begin{description}
\item[{Brief}] \leavevmode
BT tx tone open

\item[{Param}] \leavevmode\begin{itemize}
\item {} \begin{description}
\item[{en:}] \leavevmode\begin{itemize}
\item {} 
1: BT tx tone enable

\item {} 
0: BT tx tone disable

\end{itemize}

\end{description}

\item {} 
chan: BT tx channel set (0 to 78)

\item {} 
backoff: tone power attenuation set,step is 1(0.25dbm)

\end{itemize}

\item[{返回}] \leavevmode
\begin{itemize}
\item {} 
no return

\end{itemize}


\end{description}\end{quote}

\end{fulllineitems}

\index{fcc\_bt\_tx() (bt\_api.BTAPI 方法)}

\begin{fulllineitems}
\phantomsection\label{\detokenize{rfapi/index:bt_api.BTAPI.fcc_bt_tx}}\pysiglinewithargsret{\sphinxbfcode{\sphinxupquote{fcc\_bt\_tx}}}{\emph{pwr\_level}, \emph{FH\_en}, \emph{tx\_chan}, \emph{rate}, \emph{DH}, \emph{datatype}}{}~\begin{quote}\begin{description}
\item[{Brief}] \leavevmode
BR/EDR tx open

\item[{Param}] \leavevmode\begin{itemize}
\item {} 
pwr\_level: TX power level,range 0 to 7,step 3dbm

\item {} \begin{description}
\item[{FH\_en:}] \leavevmode\begin{itemize}
\item {} 
1: frequency hopping enable

\item {} 
0: frequency hopping disable

\end{itemize}

\end{description}

\item {} 
tx\_chan: BT tx channel set (0 to 78)

\item {} 
rate: tx rate set,1=1Mbps,2=2Mbps,3=3Mbps

\item {} 
DH: 1=DH1.3=DH3,5=DH5

\item {} 
datatype: 0: 01010101, 1: 00001111, 2: prbs9

\end{itemize}

\item[{返回}] \leavevmode
\begin{itemize}
\item {} 
no return

\end{itemize}


\end{description}\end{quote}

\end{fulllineitems}

\index{fcc\_le\_tx() (bt\_api.BTAPI 方法)}

\begin{fulllineitems}
\phantomsection\label{\detokenize{rfapi/index:bt_api.BTAPI.fcc_le_tx}}\pysiglinewithargsret{\sphinxbfcode{\sphinxupquote{fcc\_le\_tx}}}{\emph{pwr\_level}, \emph{tx\_chan}, \emph{payload\_len}, \emph{datatype}}{}~\begin{quote}\begin{description}
\item[{Brief}] \leavevmode
LE tx open

\item[{Param}] \leavevmode\begin{itemize}
\item {} 
pwr\_level: TX power level,range 0 to 9,step 2dbm,deafult 4

\item {} 
tx\_chan: BT tx channel set (0 to 39)

\item {} 
payload\_len: payload length,range 0 to 255,deafult 250

\item {} 
datatype: 0: 01010101, 1: 00001111, 2: prbs9

\end{itemize}

\item[{返回}] \leavevmode
\begin{itemize}
\item {} 
no return

\end{itemize}


\end{description}\end{quote}

\end{fulllineitems}

\index{fcc\_le\_tx\_syncw() (bt\_api.BTAPI 方法)}

\begin{fulllineitems}
\phantomsection\label{\detokenize{rfapi/index:bt_api.BTAPI.fcc_le_tx_syncw}}\pysiglinewithargsret{\sphinxbfcode{\sphinxupquote{fcc\_le\_tx\_syncw}}}{\emph{pwr\_level}, \emph{tx\_chan}, \emph{payload\_len}, \emph{datatype}, \emph{syncw}}{}~\begin{quote}\begin{description}
\item[{Brief}] \leavevmode
le tx (add synchronization of DC offset compensation and identification)open

\item[{Param}] \leavevmode\begin{itemize}
\item {} 
pwr\_level: TX power level,range 0 to 9,step 2dbm,deafult 4

\item {} 
tx\_chan: BT tx channel set (0 to 39)

\item {} 
payload\_len: payload length,range 0 to 255,deafult 250

\item {} 
datatype: 0: 01010101, 1: 00001111, 2: prbs9

\item {} 
syncw: deafult syncw=0x71764129

\end{itemize}

\item[{返回}] \leavevmode
\begin{itemize}
\item {} 
no return

\end{itemize}


\end{description}\end{quote}

\end{fulllineitems}

\index{rw\_le\_rx\_per() (bt\_api.BTAPI 方法)}

\begin{fulllineitems}
\phantomsection\label{\detokenize{rfapi/index:bt_api.BTAPI.rw_le_rx_per}}\pysiglinewithargsret{\sphinxbfcode{\sphinxupquote{rw\_le\_rx\_per}}}{\emph{rx\_chan}, \emph{syncw}}{}~\begin{quote}\begin{description}
\item[{Brief}] \leavevmode
LE Rx open

\item[{Param}] \leavevmode\begin{itemize}
\item {} \begin{description}
\item[{rx\_chan: rx channel set (0 to 39)}] \leavevmode\begin{itemize}
\item {} 
channel 0、1、2\textendash{}10 is corresponding frequency 2404MHz、2406MHz、2408MHz\textendash{}2424MHz

\item {} 
channel 11、12、13\textendash{}36 is corresponding frequency 2428MHz、2430MHz、2432MHz\textendash{}2478MHz

\item {} 
channel 37、38、39 is corresponding frequency 2402MHz、2426MHz、2480MHz

\end{itemize}

\end{description}

\item {} 
syncw: synchronization of DC offset compensation and identification,is decide for instrument

\end{itemize}

\item[{返回}] \leavevmode
\begin{itemize}
\item {} 
no return

\end{itemize}


\end{description}\end{quote}

\end{fulllineitems}

\index{rw\_rx\_per() (bt\_api.BTAPI 方法)}

\begin{fulllineitems}
\phantomsection\label{\detokenize{rfapi/index:bt_api.BTAPI.rw_rx_per}}\pysiglinewithargsret{\sphinxbfcode{\sphinxupquote{rw\_rx\_per}}}{\emph{modetype}, \emph{rx\_chan}, \emph{ulap}, \emph{ltaddr}}{}~\begin{quote}\begin{description}
\item[{Brief}] \leavevmode
BR/EDR Rx open

\item[{Param}] \leavevmode\begin{itemize}
\item {} \begin{description}
\item[{modetype:}] \leavevmode\begin{itemize}
\item {} 
0: BR

\item {} 
1: EDR

\end{itemize}

\end{description}

\item {} 
rx\_chan: rx channel set (0 to 78),even number channel is from 0 to 39,uneven number is from 40 to 78, for example: 1 is channel 2, 40 is channel 1

\item {} 
ulap: BT MAC,size is 32bit,include UAP(8bit) + LAP(24bit),the param is decide for instrument

\item {} 
ltaddr: logical transport address,is decide for instrument,range 0 to 7

\end{itemize}

\item[{返回}] \leavevmode
\begin{itemize}
\item {} 
no return

\end{itemize}


\end{description}\end{quote}

\end{fulllineitems}


\end{fulllineitems}



\section{RFLIB}
\label{\detokenize{rflib/index:rflib}}\label{\detokenize{rflib/index::doc}}

\subsection{pbus}
\label{\detokenize{rflib/index:module-pbus}}\label{\detokenize{rflib/index:pbus}}\index{pbus (模块)}\index{pbus (pbus 中的类)}

\begin{fulllineitems}
\phantomsection\label{\detokenize{rflib/index:pbus.pbus}}\pysiglinewithargsret{\sphinxbfcode{\sphinxupquote{class }}\sphinxcode{\sphinxupquote{pbus.}}\sphinxbfcode{\sphinxupquote{pbus}}}{\emph{comport}, \emph{chipv='ESP32'}}{}
基类:\sphinxcode{\sphinxupquote{object}}
\index{all\_pbus() (pbus.pbus 方法)}

\begin{fulllineitems}
\phantomsection\label{\detokenize{rflib/index:pbus.pbus.all_pbus}}\pysiglinewithargsret{\sphinxbfcode{\sphinxupquote{all\_pbus}}}{}{}
\end{fulllineitems}

\index{open\_rx() (pbus.pbus 方法)}

\begin{fulllineitems}
\phantomsection\label{\detokenize{rflib/index:pbus.pbus.open_rx}}\pysiglinewithargsret{\sphinxbfcode{\sphinxupquote{open\_rx}}}{}{}
\end{fulllineitems}

\index{open\_tx() (pbus.pbus 方法)}

\begin{fulllineitems}
\phantomsection\label{\detokenize{rflib/index:pbus.pbus.open_tx}}\pysiglinewithargsret{\sphinxbfcode{\sphinxupquote{open\_tx}}}{\emph{pa=95}, \emph{bb=288}}{}
\end{fulllineitems}

\index{pbus\_debugmode() (pbus.pbus 方法)}

\begin{fulllineitems}
\phantomsection\label{\detokenize{rflib/index:pbus.pbus.pbus_debugmode}}\pysiglinewithargsret{\sphinxbfcode{\sphinxupquote{pbus\_debugmode}}}{}{}
\end{fulllineitems}

\index{pbus\_rd() (pbus.pbus 方法)}

\begin{fulllineitems}
\phantomsection\label{\detokenize{rflib/index:pbus.pbus.pbus_rd}}\pysiglinewithargsret{\sphinxbfcode{\sphinxupquote{pbus\_rd}}}{\emph{pbus\_sel}, \emph{pbus\_en\_sel}}{}
\end{fulllineitems}

\index{pbus\_workmode() (pbus.pbus 方法)}

\begin{fulllineitems}
\phantomsection\label{\detokenize{rflib/index:pbus.pbus.pbus_workmode}}\pysiglinewithargsret{\sphinxbfcode{\sphinxupquote{pbus\_workmode}}}{}{}
\end{fulllineitems}

\index{pbus\_wr() (pbus.pbus 方法)}

\begin{fulllineitems}
\phantomsection\label{\detokenize{rflib/index:pbus.pbus.pbus_wr}}\pysiglinewithargsret{\sphinxbfcode{\sphinxupquote{pbus\_wr}}}{\emph{pbus\_sel}, \emph{pbus\_en\_sel}, \emph{value}}{}
\end{fulllineitems}

\index{read\_dco() (pbus.pbus 方法)}

\begin{fulllineitems}
\phantomsection\label{\detokenize{rflib/index:pbus.pbus.read_dco}}\pysiglinewithargsret{\sphinxbfcode{\sphinxupquote{read\_dco}}}{}{}
\end{fulllineitems}

\index{set\_dco() (pbus.pbus 方法)}

\begin{fulllineitems}
\phantomsection\label{\detokenize{rflib/index:pbus.pbus.set_dco}}\pysiglinewithargsret{\sphinxbfcode{\sphinxupquote{set\_dco}}}{\emph{i1=256}, \emph{q1=256}, \emph{i2=256}, \emph{q2=256}}{}
\end{fulllineitems}


\end{fulllineitems}



\subsection{rfpll}
\label{\detokenize{rflib/index:module-rfpll}}\label{\detokenize{rflib/index:rfpll}}\index{rfpll (模块)}\index{rfpll (rfpll 中的类)}

\begin{fulllineitems}
\phantomsection\label{\detokenize{rflib/index:rfpll.rfpll}}\pysiglinewithargsret{\sphinxbfcode{\sphinxupquote{class }}\sphinxcode{\sphinxupquote{rfpll.}}\sphinxbfcode{\sphinxupquote{rfpll}}}{\emph{comport}, \emph{chipv='ESP32'}}{}
基类:\sphinxcode{\sphinxupquote{object}}
\index{read\_rfpll\_reg() (rfpll.rfpll 方法)}

\begin{fulllineitems}
\phantomsection\label{\detokenize{rflib/index:rfpll.rfpll.read_rfpll_reg}}\pysiglinewithargsret{\sphinxbfcode{\sphinxupquote{read\_rfpll\_reg}}}{}{}
\end{fulllineitems}

\index{reset() (rfpll.rfpll 方法)}

\begin{fulllineitems}
\phantomsection\label{\detokenize{rflib/index:rfpll.rfpll.reset}}\pysiglinewithargsret{\sphinxbfcode{\sphinxupquote{reset}}}{}{}
\end{fulllineitems}

\index{restart\_cal() (rfpll.rfpll 方法)}

\begin{fulllineitems}
\phantomsection\label{\detokenize{rflib/index:rfpll.rfpll.restart_cal}}\pysiglinewithargsret{\sphinxbfcode{\sphinxupquote{restart\_cal}}}{}{}
\end{fulllineitems}

\index{set\_freq() (rfpll.rfpll 方法)}

\begin{fulllineitems}
\phantomsection\label{\detokenize{rflib/index:rfpll.rfpll.set_freq}}\pysiglinewithargsret{\sphinxbfcode{\sphinxupquote{set\_freq}}}{\emph{frf}, \emph{cry\_freq=40}}{}
\end{fulllineitems}

\index{set\_freq\_outband() (rfpll.rfpll 方法)}

\begin{fulllineitems}
\phantomsection\label{\detokenize{rflib/index:rfpll.rfpll.set_freq_outband}}\pysiglinewithargsret{\sphinxbfcode{\sphinxupquote{set\_freq\_outband}}}{}{}
\end{fulllineitems}


\end{fulllineitems}



\subsection{wifi\_lib}
\label{\detokenize{rflib/index:module-wifi_lib}}\label{\detokenize{rflib/index:wifi-lib}}\index{wifi\_lib (模块)}\index{WIFILIB (wifi\_lib 中的类)}

\begin{fulllineitems}
\phantomsection\label{\detokenize{rflib/index:wifi_lib.WIFILIB}}\pysiglinewithargsret{\sphinxbfcode{\sphinxupquote{class }}\sphinxcode{\sphinxupquote{wifi\_lib.}}\sphinxbfcode{\sphinxupquote{WIFILIB}}}{\emph{comport}, \emph{chipv='ESP32'}}{}
基类:\sphinxcode{\sphinxupquote{object}}
\index{GetDesirePackNum() (wifi\_lib.WIFILIB 方法)}

\begin{fulllineitems}
\phantomsection\label{\detokenize{rflib/index:wifi_lib.WIFILIB.GetDesirePackNum}}\pysiglinewithargsret{\sphinxbfcode{\sphinxupquote{GetDesirePackNum}}}{\emph{result}}{}
\end{fulllineitems}

\index{GetEntryAddr() (wifi\_lib.WIFILIB 方法)}

\begin{fulllineitems}
\phantomsection\label{\detokenize{rflib/index:wifi_lib.WIFILIB.GetEntryAddr}}\pysiglinewithargsret{\sphinxbfcode{\sphinxupquote{GetEntryAddr}}}{\emph{result}}{}
\end{fulllineitems}

\index{GetFcsErr() (wifi\_lib.WIFILIB 方法)}

\begin{fulllineitems}
\phantomsection\label{\detokenize{rflib/index:wifi_lib.WIFILIB.GetFcsErr}}\pysiglinewithargsret{\sphinxbfcode{\sphinxupquote{GetFcsErr}}}{\emph{result}}{}
\end{fulllineitems}

\index{GetFreqoff() (wifi\_lib.WIFILIB 方法)}

\begin{fulllineitems}
\phantomsection\label{\detokenize{rflib/index:wifi_lib.WIFILIB.GetFreqoff}}\pysiglinewithargsret{\sphinxbfcode{\sphinxupquote{GetFreqoff}}}{\emph{result}}{}
\end{fulllineitems}

\index{GetFrmCount() (wifi\_lib.WIFILIB 方法)}

\begin{fulllineitems}
\phantomsection\label{\detokenize{rflib/index:wifi_lib.WIFILIB.GetFrmCount}}\pysiglinewithargsret{\sphinxbfcode{\sphinxupquote{GetFrmCount}}}{\emph{result}}{}
\end{fulllineitems}

\index{GetGain() (wifi\_lib.WIFILIB 方法)}

\begin{fulllineitems}
\phantomsection\label{\detokenize{rflib/index:wifi_lib.WIFILIB.GetGain}}\pysiglinewithargsret{\sphinxbfcode{\sphinxupquote{GetGain}}}{\emph{result}}{}
\end{fulllineitems}

\index{GetGoodData() (wifi\_lib.WIFILIB 方法)}

\begin{fulllineitems}
\phantomsection\label{\detokenize{rflib/index:wifi_lib.WIFILIB.GetGoodData}}\pysiglinewithargsret{\sphinxbfcode{\sphinxupquote{GetGoodData}}}{\emph{result}}{}
\end{fulllineitems}

\index{GetGoodPackNum() (wifi\_lib.WIFILIB 方法)}

\begin{fulllineitems}
\phantomsection\label{\detokenize{rflib/index:wifi_lib.WIFILIB.GetGoodPackNum}}\pysiglinewithargsret{\sphinxbfcode{\sphinxupquote{GetGoodPackNum}}}{\emph{result}}{}
\end{fulllineitems}

\index{GetKeyMatch() (wifi\_lib.WIFILIB 方法)}

\begin{fulllineitems}
\phantomsection\label{\detokenize{rflib/index:wifi_lib.WIFILIB.GetKeyMatch}}\pysiglinewithargsret{\sphinxbfcode{\sphinxupquote{GetKeyMatch}}}{\emph{result}}{}
\end{fulllineitems}

\index{GetNoise() (wifi\_lib.WIFILIB 方法)}

\begin{fulllineitems}
\phantomsection\label{\detokenize{rflib/index:wifi_lib.WIFILIB.GetNoise}}\pysiglinewithargsret{\sphinxbfcode{\sphinxupquote{GetNoise}}}{\emph{result}}{}
\end{fulllineitems}

\index{GetRssi() (wifi\_lib.WIFILIB 方法)}

\begin{fulllineitems}
\phantomsection\label{\detokenize{rflib/index:wifi_lib.WIFILIB.GetRssi}}\pysiglinewithargsret{\sphinxbfcode{\sphinxupquote{GetRssi}}}{\emph{result}}{}
\end{fulllineitems}

\index{GetRxHung() (wifi\_lib.WIFILIB 方法)}

\begin{fulllineitems}
\phantomsection\label{\detokenize{rflib/index:wifi_lib.WIFILIB.GetRxHung}}\pysiglinewithargsret{\sphinxbfcode{\sphinxupquote{GetRxHung}}}{\emph{result}}{}
\end{fulllineitems}

\index{GetRxHung\_status() (wifi\_lib.WIFILIB 方法)}

\begin{fulllineitems}
\phantomsection\label{\detokenize{rflib/index:wifi_lib.WIFILIB.GetRxHung_status}}\pysiglinewithargsret{\sphinxbfcode{\sphinxupquote{GetRxHung\_status}}}{\emph{result}}{}
\end{fulllineitems}

\index{GetRxState() (wifi\_lib.WIFILIB 方法)}

\begin{fulllineitems}
\phantomsection\label{\detokenize{rflib/index:wifi_lib.WIFILIB.GetRxState}}\pysiglinewithargsret{\sphinxbfcode{\sphinxupquote{GetRxState}}}{\emph{result}}{}
\end{fulllineitems}

\index{GetTotErr() (wifi\_lib.WIFILIB 方法)}

\begin{fulllineitems}
\phantomsection\label{\detokenize{rflib/index:wifi_lib.WIFILIB.GetTotErr}}\pysiglinewithargsret{\sphinxbfcode{\sphinxupquote{GetTotErr}}}{\emph{result}}{}
\end{fulllineitems}

\index{GetValidCount() (wifi\_lib.WIFILIB 方法)}

\begin{fulllineitems}
\phantomsection\label{\detokenize{rflib/index:wifi_lib.WIFILIB.GetValidCount}}\pysiglinewithargsret{\sphinxbfcode{\sphinxupquote{GetValidCount}}}{\emph{result}}{}
\end{fulllineitems}

\index{cbw40m\_en() (wifi\_lib.WIFILIB 方法)}

\begin{fulllineitems}
\phantomsection\label{\detokenize{rflib/index:wifi_lib.WIFILIB.cbw40m_en}}\pysiglinewithargsret{\sphinxbfcode{\sphinxupquote{cbw40m\_en}}}{\emph{en=0}}{}
\end{fulllineitems}

\index{chan2freq() (wifi\_lib.WIFILIB 方法)}

\begin{fulllineitems}
\phantomsection\label{\detokenize{rflib/index:wifi_lib.WIFILIB.chan2freq}}\pysiglinewithargsret{\sphinxbfcode{\sphinxupquote{chan2freq}}}{\emph{chan}}{}
\end{fulllineitems}

\index{cmdstop() (wifi\_lib.WIFILIB 方法)}

\begin{fulllineitems}
\phantomsection\label{\detokenize{rflib/index:wifi_lib.WIFILIB.cmdstop}}\pysiglinewithargsret{\sphinxbfcode{\sphinxupquote{cmdstop}}}{}{}
\end{fulllineitems}

\index{esp\_rx() (wifi\_lib.WIFILIB 方法)}

\begin{fulllineitems}
\phantomsection\label{\detokenize{rflib/index:wifi_lib.WIFILIB.esp_rx}}\pysiglinewithargsret{\sphinxbfcode{\sphinxupquote{esp\_rx}}}{\emph{chan}, \emph{rate\_sym}}{}
\end{fulllineitems}

\index{esp\_tx() (wifi\_lib.WIFILIB 方法)}

\begin{fulllineitems}
\phantomsection\label{\detokenize{rflib/index:wifi_lib.WIFILIB.esp_tx}}\pysiglinewithargsret{\sphinxbfcode{\sphinxupquote{esp\_tx}}}{\emph{chan=1}, \emph{ratenum=23}, \emph{backoff=0}}{}
\end{fulllineitems}

\index{force\_tx\_gain\_init() (wifi\_lib.WIFILIB 方法)}

\begin{fulllineitems}
\phantomsection\label{\detokenize{rflib/index:wifi_lib.WIFILIB.force_tx_gain_init}}\pysiglinewithargsret{\sphinxbfcode{\sphinxupquote{force\_tx\_gain\_init}}}{\emph{chan=1}, \emph{rate='mcs7'}, \emph{pa\_gain=95}, \emph{bb\_gain=288}, \emph{dig\_gain=244}}{}
\end{fulllineitems}

\index{force\_txon() (wifi\_lib.WIFILIB 方法)}

\begin{fulllineitems}
\phantomsection\label{\detokenize{rflib/index:wifi_lib.WIFILIB.force_txon}}\pysiglinewithargsret{\sphinxbfcode{\sphinxupquote{force\_txon}}}{\emph{en=0}}{}
\end{fulllineitems}

\index{get\_filename() (wifi\_lib.WIFILIB 方法)}

\begin{fulllineitems}
\phantomsection\label{\detokenize{rflib/index:wifi_lib.WIFILIB.get_filename}}\pysiglinewithargsret{\sphinxbfcode{\sphinxupquote{get\_filename}}}{\emph{folder}, \emph{file\_name}, \emph{sub\_folder=''}}{}~\begin{quote}\begin{description}
\item[{Folder}] \leavevmode
file store folder

\item[{File\_name}] \leavevmode
file name

\item[{Sub\_folder}] \leavevmode
if not need, it may be default “”

\end{description}\end{quote}

\end{fulllineitems}

\index{get\_length\_delay\_duty() (wifi\_lib.WIFILIB 方法)}

\begin{fulllineitems}
\phantomsection\label{\detokenize{rflib/index:wifi_lib.WIFILIB.get_length_delay_duty}}\pysiglinewithargsret{\sphinxbfcode{\sphinxupquote{get\_length\_delay\_duty}}}{\emph{rate=23}, \emph{duty=0.1}}{}
set the duty cycle of tx packet,duty range(0,1{]}

\end{fulllineitems}

\index{get\_length\_delay\_mean() (wifi\_lib.WIFILIB 方法)}

\begin{fulllineitems}
\phantomsection\label{\detokenize{rflib/index:wifi_lib.WIFILIB.get_length_delay_mean}}\pysiglinewithargsret{\sphinxbfcode{\sphinxupquote{get\_length\_delay\_mean}}}{\emph{rate=23}}{}
set the diffrent duty cycle to meet average current at 140mA

\end{fulllineitems}

\index{get\_ratelst() (wifi\_lib.WIFILIB 方法)}

\begin{fulllineitems}
\phantomsection\label{\detokenize{rflib/index:wifi_lib.WIFILIB.get_ratelst}}\pysiglinewithargsret{\sphinxbfcode{\sphinxupquote{get\_ratelst}}}{\emph{ratelst}, \emph{rx\_rate\_option=''}}{}
\end{fulllineitems}

\index{get\_rx\_cable\_lost() (wifi\_lib.WIFILIB 方法)}

\begin{fulllineitems}
\phantomsection\label{\detokenize{rflib/index:wifi_lib.WIFILIB.get_rx_cable_lost}}\pysiglinewithargsret{\sphinxbfcode{\sphinxupquote{get\_rx\_cable\_lost}}}{\emph{iqv\_unit\_no=1, cable\_att=30, chan\_m={[}14{]}, noise\_ref=-95.2}}{}
\end{fulllineitems}

\index{get\_rx\_tone\_pwr() (wifi\_lib.WIFILIB 方法)}

\begin{fulllineitems}
\phantomsection\label{\detokenize{rflib/index:wifi_lib.WIFILIB.get_rx_tone_pwr}}\pysiglinewithargsret{\sphinxbfcode{\sphinxupquote{get\_rx\_tone\_pwr}}}{\emph{rx\_freq\_mhz=5}, \emph{rx\_freq\_cmp\_mult20=0}, \emph{gain\_force=0}, \emph{gain=40}}{}
\end{fulllineitems}

\index{get\_rx\_tone\_pwr\_scan() (wifi\_lib.WIFILIB 方法)}

\begin{fulllineitems}
\phantomsection\label{\detokenize{rflib/index:wifi_lib.WIFILIB.get_rx_tone_pwr_scan}}\pysiglinewithargsret{\sphinxbfcode{\sphinxupquote{get\_rx\_tone\_pwr\_scan}}}{\emph{rx\_freq\_mhz=5}, \emph{gain\_force=0}, \emph{gain=40}, \emph{scan\_range=10}}{}
\end{fulllineitems}

\index{i2c\_ric() (wifi\_lib.WIFILIB 方法)}

\begin{fulllineitems}
\phantomsection\label{\detokenize{rflib/index:wifi_lib.WIFILIB.i2c_ric}}\pysiglinewithargsret{\sphinxbfcode{\sphinxupquote{i2c\_ric}}}{\emph{block}, \emph{ctrl\_name}}{}
\end{fulllineitems}

\index{i2c\_wic() (wifi\_lib.WIFILIB 方法)}

\begin{fulllineitems}
\phantomsection\label{\detokenize{rflib/index:wifi_lib.WIFILIB.i2c_wic}}\pysiglinewithargsret{\sphinxbfcode{\sphinxupquote{i2c\_wic}}}{\emph{block}, \emph{ctrl\_name}, \emph{value}}{}
\end{fulllineitems}

\index{rate2ht40() (wifi\_lib.WIFILIB 方法)}

\begin{fulllineitems}
\phantomsection\label{\detokenize{rflib/index:wifi_lib.WIFILIB.rate2ht40}}\pysiglinewithargsret{\sphinxbfcode{\sphinxupquote{rate2ht40}}}{\emph{rate}}{}
\end{fulllineitems}

\index{ratecheck() (wifi\_lib.WIFILIB 方法)}

\begin{fulllineitems}
\phantomsection\label{\detokenize{rflib/index:wifi_lib.WIFILIB.ratecheck}}\pysiglinewithargsret{\sphinxbfcode{\sphinxupquote{ratecheck}}}{\emph{rate\_sym}}{}
\end{fulllineitems}

\index{read\_mac() (wifi\_lib.WIFILIB 方法)}

\begin{fulllineitems}
\phantomsection\label{\detokenize{rflib/index:wifi_lib.WIFILIB.read_mac}}\pysiglinewithargsret{\sphinxbfcode{\sphinxupquote{read\_mac}}}{}{}
\end{fulllineitems}

\index{rf\_tune() (wifi\_lib.WIFILIB 方法)}

\begin{fulllineitems}
\phantomsection\label{\detokenize{rflib/index:wifi_lib.WIFILIB.rf_tune}}\pysiglinewithargsret{\sphinxbfcode{\sphinxupquote{rf\_tune}}}{}{}
\end{fulllineitems}

\index{rfchsel() (wifi\_lib.WIFILIB 方法)}

\begin{fulllineitems}
\phantomsection\label{\detokenize{rflib/index:wifi_lib.WIFILIB.rfchsel}}\pysiglinewithargsret{\sphinxbfcode{\sphinxupquote{rfchsel}}}{\emph{chan}, \emph{cbw2040\_cfg=0}}{}
\end{fulllineitems}

\index{rxdc\_cal() (wifi\_lib.WIFILIB 方法)}

\begin{fulllineitems}
\phantomsection\label{\detokenize{rflib/index:wifi_lib.WIFILIB.rxdc_cal}}\pysiglinewithargsret{\sphinxbfcode{\sphinxupquote{rxdc\_cal}}}{}{}
\end{fulllineitems}

\index{rxresult() (wifi\_lib.WIFILIB 方法)}

\begin{fulllineitems}
\phantomsection\label{\detokenize{rflib/index:wifi_lib.WIFILIB.rxresult}}\pysiglinewithargsret{\sphinxbfcode{\sphinxupquote{rxresult}}}{\emph{result\_data}}{}
\end{fulllineitems}

\index{rxstart() (wifi\_lib.WIFILIB 方法)}

\begin{fulllineitems}
\phantomsection\label{\detokenize{rflib/index:wifi_lib.WIFILIB.rxstart}}\pysiglinewithargsret{\sphinxbfcode{\sphinxupquote{rxstart}}}{\emph{rate\_sym}}{}
\end{fulllineitems}

\index{save\_init\_print() (wifi\_lib.WIFILIB 方法)}

\begin{fulllineitems}
\phantomsection\label{\detokenize{rflib/index:wifi_lib.WIFILIB.save_init_print}}\pysiglinewithargsret{\sphinxbfcode{\sphinxupquote{save\_init\_print}}}{\emph{folder=''}}{}
\end{fulllineitems}

\index{stoptone() (wifi\_lib.WIFILIB 方法)}

\begin{fulllineitems}
\phantomsection\label{\detokenize{rflib/index:wifi_lib.WIFILIB.stoptone}}\pysiglinewithargsret{\sphinxbfcode{\sphinxupquote{stoptone}}}{}{}
\end{fulllineitems}

\index{test\_txtone\_pwr() (wifi\_lib.WIFILIB 方法)}

\begin{fulllineitems}
\phantomsection\label{\detokenize{rflib/index:wifi_lib.WIFILIB.test_txtone_pwr}}\pysiglinewithargsret{\sphinxbfcode{\sphinxupquote{test\_txtone\_pwr}}}{\emph{atten}, \emph{loop\_num}, \emph{mode=0}, \emph{step=0}, \emph{delay\_us=10}}{}
\end{fulllineitems}

\index{tx\_cbw40m\_en() (wifi\_lib.WIFILIB 方法)}

\begin{fulllineitems}
\phantomsection\label{\detokenize{rflib/index:wifi_lib.WIFILIB.tx_cbw40m_en}}\pysiglinewithargsret{\sphinxbfcode{\sphinxupquote{tx\_cbw40m\_en}}}{\emph{en=0}}{}
\end{fulllineitems}

\index{tx\_contin\_en() (wifi\_lib.WIFILIB 方法)}

\begin{fulllineitems}
\phantomsection\label{\detokenize{rflib/index:wifi_lib.WIFILIB.tx_contin_en}}\pysiglinewithargsret{\sphinxbfcode{\sphinxupquote{tx\_contin\_en}}}{\emph{en=0}}{}
\end{fulllineitems}

\index{txout() (wifi\_lib.WIFILIB 方法)}

\begin{fulllineitems}
\phantomsection\label{\detokenize{rflib/index:wifi_lib.WIFILIB.txout}}\pysiglinewithargsret{\sphinxbfcode{\sphinxupquote{txout}}}{\emph{rate\_sym}, \emph{PackNum=0}, \emph{PackLen=1024}, \emph{cbw40=0}, \emph{ht\_dup=0}, \emph{backoff\_qdb=0}, \emph{frm\_delay=2000}}{}
\end{fulllineitems}

\index{txpacket() (wifi\_lib.WIFILIB 方法)}

\begin{fulllineitems}
\phantomsection\label{\detokenize{rflib/index:wifi_lib.WIFILIB.txpacket}}\pysiglinewithargsret{\sphinxbfcode{\sphinxupquote{txpacket}}}{\emph{txchan=1}, \emph{rate\_sym='mcs7'}, \emph{PackNum=0}, \emph{cbw40=0}, \emph{ht\_dup=0}, \emph{backoff\_qdb=0}, \emph{duty=0}}{}
\end{fulllineitems}

\index{txtone() (wifi\_lib.WIFILIB 方法)}

\begin{fulllineitems}
\phantomsection\label{\detokenize{rflib/index:wifi_lib.WIFILIB.txtone}}\pysiglinewithargsret{\sphinxbfcode{\sphinxupquote{txtone}}}{\emph{tone1\_en=1}, \emph{freq1\_mhz=2}, \emph{tone1\_att=0}, \emph{tone2\_en=0}, \emph{freq2\_mhz=0}, \emph{tone2\_att=0}}{}
\end{fulllineitems}

\index{wifiscwout() (wifi\_lib.WIFILIB 方法)}

\begin{fulllineitems}
\phantomsection\label{\detokenize{rflib/index:wifi_lib.WIFILIB.wifiscwout}}\pysiglinewithargsret{\sphinxbfcode{\sphinxupquote{wifiscwout}}}{\emph{en=0}, \emph{chan=1}, \emph{backoff=0}}{}
\end{fulllineitems}


\end{fulllineitems}



\subsection{adc\_dump}
\label{\detokenize{rflib/index:module-adc_dump}}\label{\detokenize{rflib/index:adc-dump}}\index{adc\_dump (模块)}\index{DUMP (adc\_dump 中的类)}

\begin{fulllineitems}
\phantomsection\label{\detokenize{rflib/index:adc_dump.DUMP}}\pysiglinewithargsret{\sphinxbfcode{\sphinxupquote{class }}\sphinxcode{\sphinxupquote{adc\_dump.}}\sphinxbfcode{\sphinxupquote{DUMP}}}{\emph{comport}, \emph{chipv='ESP32'}}{}
基类:\sphinxcode{\sphinxupquote{object}}
\index{adcdump() (adc\_dump.DUMP 方法)}

\begin{fulllineitems}
\phantomsection\label{\detokenize{rflib/index:adc_dump.DUMP.adcdump}}\pysiglinewithargsret{\sphinxbfcode{\sphinxupquote{adcdump}}}{\emph{logdir}, \emph{start\_addr}, \emph{byte\_len}, \emph{burst\_len}, \emph{buff\_start}, \emph{buff\_size}, \emph{trig\_pos}, \emph{adc\_version='10bit'}, \emph{chan\_no=1}, \emph{dump\_13bit=0}}{}
\end{fulllineitems}

\index{adcdumptest() (adc\_dump.DUMP 方法)}

\begin{fulllineitems}
\phantomsection\label{\detokenize{rflib/index:adc_dump.DUMP.adcdumptest}}\pysiglinewithargsret{\sphinxbfcode{\sphinxupquote{adcdumptest}}}{\emph{logdir='dump'}, \emph{dump\_num=1024}, \emph{trig\_mode='sw'}, \emph{adc\_version='10bit'}, \emph{sample\_80m=0}, \emph{plot\_en=0}, \emph{chan\_en=0}, \emph{chan=14}, \emph{force\_gain\_en=0}, \emph{force\_gain=70}, \emph{force\_gain0=70}, \emph{gain0\_wait=0}, \emph{rxgain\_offset=40}, \emph{trigcase=0}, \emph{dump\_trig=0}, \emph{subplot\_en=0}, \emph{subplot\_row=3}, \emph{subplot\_col=3}, \emph{index=0}, \emph{plot\_save=0}, \emph{dump\_13bit=0}}{}
dump\_trig: 1 is dump first, then trig, 0 is trig first, then dump.

\end{fulllineitems}

\index{get\_dump\_accm() (adc\_dump.DUMP 方法)}

\begin{fulllineitems}
\phantomsection\label{\detokenize{rflib/index:adc_dump.DUMP.get_dump_accm}}\pysiglinewithargsret{\sphinxbfcode{\sphinxupquote{get\_dump\_accm}}}{\emph{accum\_i}, \emph{accum\_q}, \emph{rssi}, \emph{mem\_addr}, \emph{burst\_len}}{}
\end{fulllineitems}

\index{get\_dump\_data() (adc\_dump.DUMP 方法)}

\begin{fulllineitems}
\phantomsection\label{\detokenize{rflib/index:adc_dump.DUMP.get_dump_data}}\pysiglinewithargsret{\sphinxbfcode{\sphinxupquote{get\_dump\_data}}}{\emph{dump\_13bit}, \emph{adc\_data}}{}
\end{fulllineitems}

\index{sampledeal() (adc\_dump.DUMP 方法)}

\begin{fulllineitems}
\phantomsection\label{\detokenize{rflib/index:adc_dump.DUMP.sampledeal}}\pysiglinewithargsret{\sphinxbfcode{\sphinxupquote{sampledeal}}}{\emph{sample}, \emph{adc\_version='8bit'}}{}
\end{fulllineitems}

\index{set\_dump\_mode() (adc\_dump.DUMP 方法)}

\begin{fulllineitems}
\phantomsection\label{\detokenize{rflib/index:adc_dump.DUMP.set_dump_mode}}\pysiglinewithargsret{\sphinxbfcode{\sphinxupquote{set\_dump\_mode}}}{\emph{dump\_13bit=0}}{}
\end{fulllineitems}

\index{write\_dump\_data() (adc\_dump.DUMP 方法)}

\begin{fulllineitems}
\phantomsection\label{\detokenize{rflib/index:adc_dump.DUMP.write_dump_data}}\pysiglinewithargsret{\sphinxbfcode{\sphinxupquote{write\_dump\_data}}}{\emph{accum\_i}, \emph{accum\_q}, \emph{accum\_lna\_sat}, \emph{accum\_vga\_sat}, \emph{rssi}, \emph{mem\_addr}, \emph{burst\_len}, \emph{csvwrite}, \emph{dump\_13bit=0}}{}
\end{fulllineitems}


\end{fulllineitems}



\section{TESTCASE}
\label{\detokenize{testcase/index:testcase}}\label{\detokenize{testcase/index::doc}}

\subsection{rf\_debug}
\label{\detokenize{testcase/index:rf-debug}}\begin{quote}

Store RF submodule test scripts

for example: PA, I2C, TX Gain, RX Gain test ….
\end{quote}


\subsection{performance}
\label{\detokenize{testcase/index:performance}}\begin{quote}

Store RF performance scripts

for example: wifiTX WIFIRX BTTX BTRX performance …
\end{quote}


\subsection{current}
\label{\detokenize{testcase/index:current}}\begin{quote}

Store TX/RX current measurement scripts
\end{quote}


\subsection{temprature}
\label{\detokenize{testcase/index:temprature}}\begin{quote}

Store multiple chips testing scripts in temperature chamber
\end{quote}


\section{TestReport}
\label{\detokenize{testreport/index:testreport}}\label{\detokenize{testreport/index::doc}}

\renewcommand{\indexname}{Python 模块索引}
\begin{sphinxtheindex}
\let\bigletter\sphinxstyleindexlettergroup
\bigletter{a}
\item\relax\sphinxstyleindexentry{adc\_dump}\sphinxstyleindexpageref{rflib/index:\detokenize{module-adc_dump}}
\indexspace
\bigletter{b}
\item\relax\sphinxstyleindexentry{bt\_api}\sphinxstyleindexpageref{rfapi/index:\detokenize{module-bt_api}}
\indexspace
\bigletter{c}
\item\relax\sphinxstyleindexentry{common}\sphinxstyleindexpageref{rfapi/index:\detokenize{module-common}}
\indexspace
\bigletter{p}
\item\relax\sphinxstyleindexentry{pbus}\sphinxstyleindexpageref{rflib/index:\detokenize{module-pbus}}
\indexspace
\bigletter{r}
\item\relax\sphinxstyleindexentry{rfpll}\sphinxstyleindexpageref{rflib/index:\detokenize{module-rfpll}}
\indexspace
\bigletter{w}
\item\relax\sphinxstyleindexentry{wifi\_api}\sphinxstyleindexpageref{rfapi/index:\detokenize{module-wifi_api}}
\item\relax\sphinxstyleindexentry{wifi\_lib}\sphinxstyleindexpageref{rflib/index:\detokenize{module-wifi_lib}}
\end{sphinxtheindex}

\renewcommand{\indexname}{索引}
\printindex
\end{document}